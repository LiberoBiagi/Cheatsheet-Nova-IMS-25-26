\section{Programming}
\subsection{Theory and code examples}

\subsubsection{Introduction to Python and computers}

\textbf{Printing} \ra A string is returned and shown on the screen
\begin{pythoncode}
    print("Hello World")
\end{pythoncode}

\textbf{print() syntax} \ra print(object(s), separator=separator, end=end, file=file, flush=flush)

\vspace{20pt}

\textbf{Errors} \ra They happen when there is a problem in the code

\begin{itemize}
    \item \textbf{Runtime Errors} \ra Something wrong, the code won't run 
    \item \textbf{Semantic Errors} \ra The output is not what we expected
    
\end{itemize}

We have many more of them, to solve we have to \textbf{debug the code}.

\vspace{20pt}

Using Jupyter notebooks we can use some special commands that are preceded by the \% character
\begin{itemize}
    \item \textbf{\%timeit} \ra determines the execution time of a single-line
statement. Performs multiple runs to achieve robust results.
    \item \textbf{\%\%timeit} \ra same as \%timeit but for the entire cell
    \item \textbf{\%time}\ra determines the execution time of a single-line
statement. Performs a single run!
    \item \textbf{\%\%time}\ra same as \%time but for the entire cell
    \item \textbf{\%run}\ra Run the named file inside IPython as a program
    \item \textbf{\%history}\ra displays the command history. Use \%history -n to display last n-commands with line
numbers.
    \item \textbf{\%recall<line\_no>}\ra  re-executes command at line\_no. You can also specify range of line numbers
    \item \textbf{\%who}\ra Shows list of all variables defined within the current notebook
    \item \textbf{\%Ismagic}\ra Shows a list of magic commands
    \item \textbf{\%magic}\ra Quick and simple list of all available magic functions with detailed descriptions
    \item \textbf{\%quickref}\ra List of common magic commands and their descriptions
\end{itemize}

And many more

\vspace{10pt}

\hrule

\subsubsection{Semantics, datatypes, flow controls and structures}

\vspace{10pt}

\textbf{Semantics} \ra in Python tabs and spaces are used to structure the code. If you use a colon you have to indent the code in the right way. Semicolons instead can be used for multiple statements on the same line

\vspace{10pt}

\textbf{Objects} \ra everything in Python is an object with it's own methods, functions and characteristics

\vspace{10pt}

\textbf{Comments} \ra \# denotes comments

\vspace{10pt}

To represent information we can use different kind of data types and structures

\vspace{10pt}

\textbf{Variables} \ra a variable can take whatever value we want

\begin{pythoncode}
    a = 2 #variable declaration
    print(a)
>>> 2
    type(a)
>>>int
\end{pythoncode}

\vspace{10pt}

Other possible types are

\begin{itemize}
    \item float
    \item string
    \item complex
    \item boolean
\end{itemize}

\vspace{10pt}

If we want to save a collection of objects we can use

\begin{pythoncode}
    List = [1, 3, "a", 7]
    Tuple = (1, "a", 8, 7)
    Dictionary = {"a":1, "b":4}
    Set = {1, "a", 4, 8}
\end{pythoncode}

\begin{table}[h!]
\centering
\renewcommand{\arraystretch}{1.3}
\begin{tabular}{|c|c|c|c|}
\hline
\textbf{Data Structure} & \textbf{Mutable / Immutable} & \textbf{Ordered / Unordered} & \textbf{Indexed / History} \\
\hline
List        & Mutable   & Ordered   & Indexed \\
Dictionary  & Mutable   & Unordered & History of addition \\
Tuple       & Immutable & Ordered   & Indexed \\
Set         & Mutable   & Unordered & No indexing (unique elements) \\
\hline
\end{tabular}
\caption{Comparison of Python collection data structures.}
\end{table}

\vspace{10pt}

Every data structure has it's own methods.

When we are using indexed data structures we have to remember that the first index is 0.

\vspace{10pt}

Since we want to work with data and modify them we need a way to do it. Operators can take variables and make computations if those are compatible

\begin{table}[h!]
\centering
\renewcommand{\arraystretch}{1.3}
\begin{tabular}{|c|c|c|}
\hline
\textbf{Operator} & \textbf{Name} & \textbf{Example} \\
\hline
\texttt{+}  & Addition        & \texttt{x + y}  \\
\texttt{-}  & Subtraction     & \texttt{x - y}  \\
\texttt{*}  & Multiplication  & \texttt{x * y}  \\
\texttt{/}  & Division        & \texttt{x / y}  \\
\texttt{\%} & Modulus         & \texttt{x \% y} \\
\texttt{**} & Exponentiation  & \texttt{x ** y} \\
\texttt{//} & Floor division  & \texttt{x // y} \\
\hline
\end{tabular}
\caption{Python arithmetic operators with names and examples.}
\end{table}
\begin{table}[H]
\centering
\renewcommand{\arraystretch}{1.3}
\begin{tabular}{|c|c|c|}
\hline
\textbf{Operator} & \textbf{Example} & \textbf{Same As} \\
\hline
\texttt{=}   & \texttt{x = 5}    & \texttt{x = 5} \\
\texttt{+=}  & \texttt{x += 3}   & \texttt{x = x + 3} \\
\texttt{-=}  & \texttt{x -= 3}   & \texttt{x = x - 3} \\
\texttt{*=}  & \texttt{x *= 3}   & \texttt{x = x * 3} \\
\texttt{/=}  & \texttt{x /= 3}   & \texttt{x = x / 3} \\
\texttt{\%=} & \texttt{x \%= 3}  & \texttt{x = x \% 3} \\
\texttt{//=} & \texttt{x //= 3}  & \texttt{x = x // 3} \\
\texttt{**=} & \texttt{x **= 3}  & \texttt{x = x ** 3} \\
\texttt{\&=} & \texttt{x \&= 3}  & \texttt{x = x \& 3} \\
\texttt{|=}  & \texttt{x |= 3}   & \texttt{x = x | 3} \\
\texttt{\^=} & \texttt{x \^= 3}  & \texttt{x = x \^ 3} \\
\texttt{>>=} & \texttt{x >>= 3}  & \texttt{x = x >> 3} \\
\texttt{<<=} & \texttt{x <<= 3}  & \texttt{x = x << 3} \\
\hline
\end{tabular}
\caption{Python assignment operators with examples.}
\end{table}

\begin{table}[h!]
\centering
\renewcommand{\arraystretch}{1.3}
\begin{tabular}{|c|c|c|}
\hline
\textbf{Operator} & \textbf{Name} & \textbf{Example} \\
\hline
\texttt{==} & Equal                    & \texttt{x == y} \\
\texttt{!=} & Not equal                & \texttt{x != y} \\
\texttt{>}  & Greater than             & \texttt{x > y}  \\
\texttt{<}  & Less than                & \texttt{x < y}  \\
\texttt{>=} & Greater than or equal to & \texttt{x >= y} \\
\texttt{<=} & Less than or equal to    & \texttt{x <= y} \\
\hline
\end{tabular}
\caption{Python comparison operators.}
\end{table}

\begin{table}[h!]
\centering
\renewcommand{\arraystretch}{1.3}
\begin{tabular}{|c|c|c|}
\hline
\textbf{Operator} & \textbf{Description} & \textbf{Example} \\
\hline
\texttt{and} & Returns True if both statements are true & \texttt{x < 5 and x < 10} \\
\texttt{or}  & Returns True if one of the statements is true & \texttt{x < 5 or x < 4} \\
\texttt{not} & Reverses the result, returns False if the result is True & \texttt{not(x < 5 and x < 10)} \\
\hline
\end{tabular}
\caption{Python logical operators.}
\end{table}

\begin{table}[h!]
\centering
\renewcommand{\arraystretch}{1.3}
\begin{tabular}{|c|c|p{8cm}|}
\hline
\textbf{Operator} & \textbf{Name} & \textbf{Description} \\
\hline
\texttt{\&}  & AND  & Sets each bit to 1 if both bits are 1 \\
\texttt{|}   & OR   & Sets each bit to 1 if one of two bits is 1 \\
\texttt{\^}  & XOR  & Sets each bit to 1 if only one of two bits is 1 \\
\texttt{\~}  & NOT  & Inverts all the bits \\
\texttt{<<}  & Zero fill left shift  & Shift left by pushing zeros in from the right and let the leftmost bits fall off \\
\texttt{>>}  & Signed right shift   & Shift right by pushing copies of the leftmost bit in from the left, and let the rightmost bits fall off \\
\hline
\end{tabular}
\caption{Python bitwise operators.}
\end{table}

\begin{table}[H]
\centering
\renewcommand{\arraystretch}{1.3}
\begin{tabular}{|c|p{10cm}|}
\hline
\textbf{Operator} & \textbf{Description} \\
\hline
\texttt{is}     & Returns True if both variables are the same object \\
\texttt{is not} & Returns True if both variables are not the same object \\
\hline
\end{tabular}
\caption{Python identity operators.}
\end{table}

\vspace{10pt}

\textbf{Flow control} \ra to check that our program is doing what we want we can apply different statements.

\textbf{if} \ra we define a condition under which a certain action is done if that condition is reached

\begin{table}[h!]
\centering
\renewcommand{\arraystretch}{1.3}
\begin{tabular}{|c|c|c|}
\hline
\textbf{Operator} & \textbf{Name} & \textbf{Example} \\
\hline
\texttt{==} & Equal                    & \texttt{x == y} \\
\texttt{!=} & Not equal                & \texttt{x != y} \\
\texttt{>}  & Greater than             & \texttt{x > y}  \\
\texttt{<}  & Less than                & \texttt{x < y}  \\
\texttt{>=} & Greater than or equal to & \texttt{x >= y} \\
\texttt{<=} & Less than or equal to    & \texttt{x <= y} \\
\hline
\end{tabular}
\caption{Python comparison operators with examples for conditional statements.}
\end{table}

\textbf{elif} \ra it means else if, we specify a new condition

\textbf{else} \ra to be put as last, if neither if or elif are met we use else

\begin{pythoncode}
if name == "Libero":
    print("Hello")
elif name == "Jose":
    print("Forza Milan")
else:
    print("None")
\end{pythoncode}


\vspace{10pt}

\textbf{Loops} \ra for now we just cared about one time operations, we can do those on multiple times with loops. 

We have two main loops

\begin{itemize}
    \item \textbf{for loops} \ra we specify an interval and the action is performed that number of times.
    \item \textbf{while loops} \ra the operation is made as long as a certain condition is true
\end{itemize}

\begin{pythoncode}
    for x in range (10):
        print(x)

    a = 0
    while a < 25:
        print(a)
        a +=1
\end{pythoncode}

Some problems can arise if the condition of the while loop is never met.

\newpage



\textbf{Comprehension} \ra for faster performances we can use this particular structure. It works for all the mutable data structure

\begin{pythoncode}
    squares = [x**2 for x in range(10)]
    print(squares)

    even_squares = [x**2 for x in range(10) if x % 2 == 0]
    print(even_squares)


    squares_dict = {x: x**2 for x in range(5)}
    print(squares_dict)

    squares_set = {x**2 for x in range(5)}
    print(squares_set)
\end{pythoncode}

With this general structure

\begin{pythoncode}
    {key_expression: value_expression for item in iterable if condition}

    {expression for item in iterable if condition}

\end{pythoncode}

\textbf{Slices} \ra if we want a subset of the data structure.

\vspace{10pt}

General structure

\begin{pythoncode}
    list[start:stop:step]
\end{pythoncode}

Negative indexes mean to start from the end

\vspace{10pt}

\textbf{typecast} \ra we can transform a variable with a certain structure into another with another structure

\begin{pythoncode}
    # Typecasting examples in one snippet

# String to int
x_str = "10"
x_int = int(x_str)
print(x_int, type(x_int))  # 10 <class 'int'>

# String to float
y_str = "3.14"
y_float = float(y_str)
print(y_float, type(y_float))  # 3.14 <class 'float'>

# Int to string
z_int = 100
z_str = str(z_int)
print(z_str, type(z_str))  # '100' <class 'str'>

# Tuple to list
tup = (1, 2, 3)
lst = list(tup)
print(lst, type(lst))  # [1, 2, 3] <class 'list'>

# List to tuple
lst2 = [4, 5, 6]
tup2 = tuple(lst2)
print(tup2, type(tup2))  # (4, 5, 6) <class 'tuple'>

# Int to float
num = 7
num_float = float(num)
print(num_float, type(num_float))  # 7.0 <class 'float'>

# Float to int
num2 = 9.8
num2_int = int(num2)  # truncates decimal part
print(num2_int, type(num2_int))  # 9 <class 'int'>

\end{pythoncode}

\hrule

\subsubsection{Functions, imports and namespace}

\textbf{Functions} \ra like a mathematical function a python one will take some inputs and will apply the same operations to get a certain result. We define them to avoid writing the same code over and over.

\begin{pythoncode}
    def function(input): #to define
        input operation
        return output

    function(other_input) #to call
\end{pythoncode}

\vspace{10pt}

A function can also work with the elements of a list or other collections of elements using a for loop.

\vspace{10pt}

If we want to have a flexible number of inputs we can use *args and **kwargs

\begin{itemize}
    \item \textbf{*args} \ra the function will take a list as input and will do the computations on each element of that list
    \item \textbf{**kwargs} \ra same as args but with a list of keys and values, the function will return a dictionary
\end{itemize}

\begin{pythoncode}
    def multiply(*args):
        output = 1
        for n in args:
            output *= n
        return output

    print(multiply(2, 3, 4))

    
\end{pythoncode}

\begin{pythoncode}
    def my_function(**kwargs):
        for key, value in kwargs.items():
            print(key, value)

    my_function(name="Alice", age=25, city="Paris")

    ###Output
name Alice
age 25
city Paris
\end{pythoncode}

\href{https://youtu.be/4jBJhCaNrWU?si=h0v33a0kZZ5CMMfV}{Link for a video if it's still confusing}

\vspace{10pt}

You can add infos about your function with the \_\_doc\_\_

\begin{pythoncode}
    def greets(*args):
        """This function says Hello"""
        for name in args:
            print("Hello", name)
    greet.__doc__
\end{pythoncode}

\vspace{10pt}

If we want a fast singular use function we can use \textbf{lambda functions}.

\begin{pythoncode}
    lambda arguments: expression

    add = lambda x, y : x + y
    add(2, 3)
\end{pythoncode}

\vspace{10pt}

\textbf{Map} \ra we want to transform certain values into others in a fast way

\begin{pythoncode}
    list(map(lambda x : x * 2, [1, 2, 3, 4]))
\end{pythoncode}

\vspace{10pt}

\textbf{Filter} \ra if we want to select a subset of our data

\begin{pythoncode}
    from random import sample

    X = sample(range(-25, 25), 25)
    f = filter(lambda x: x > 0, X)
\end{pythoncode}

\vspace{10pt}

When we want to call a certain function multiple times we can use two main ways:

\begin{itemize}
    \item \textbf{Iteration} \ra We define a number of times and the function will do its job for that number.

    \item \textbf{Recursion} \ra The function will call itself until it reaches a base case. It's important to define base, edge and general cases. Useful when we work with trees or graphs
    
\end{itemize}

\begin{pythoncode}
    def factorial(n):
    if n == 1:
        return 1   #base case
    elif n == 0:
        return 1   #edge case
    else:
        return n * factorial (n-1) #general case
\end{pythoncode}

\vspace{10pt}

We should differentiate between:

\begin{itemize}
    \item \textbf{Functions} \ra defined by def or lambda, they can be applied to almost everything if well defined.
    \item \textbf{Methods} \ra associated with certain objects
    \item \textbf{Attributes} \ra variables associated to certain objects
\end{itemize}

\vspace{10pt}

Remember kids, if you want to code something probably someone has already done it. So why bother?

We can import the work done by other people

\begin{pythoncode}
    import module as mod
    from module import method1, method2

\end{pythoncode}

sys.path will give us all the directories that our interpreter is watching

Useful modules can be:
\begin{itemize}
    \item csv
    \item datetime
    \item io
    \item json
    \item math
    \item os
    \item random
    \item sqlite3
    \item xml
    \item zipfile
    \item zlib
\end{itemize}

\textbf{Namespace}  
A namespace is a collection of names that reference objects.

- \textbf{Built-in Names}: predefined names available in every Python interpreter.  
Examples: \texttt{list}, \texttt{dict}, \texttt{map}, \texttt{tuple}.
\begin{pythoncode}
import builtins
print(dir(builtins))   # list built-in names
\end{pythoncode}

- \textbf{Global Names}: user-defined names created in the main program body (variables, functions, classes).
\begin{pythoncode}
x = 42   # global variable
def foo():
    return x
print(globals())   # list global names
\end{pythoncode}

- \textbf{Local Names}: names defined inside a function, valid only inside it.
\begin{pythoncode}
def bar():
    y = 10  # local variable
    print(locals())   # list local names
bar()
# print(y)  -> Error: y does not exist in the global scope
\end{pythoncode}

\textbf{Scope}  
Scope defines where a variable can be accessed.

- \textbf{Global Scope}: variable accessible throughout the whole program.  
- \textbf{Local Scope}: variable accessible only inside the function where it was declared.

\begin{pythoncode}
animal = "dog"  # global variable

def test_scope():
    # local variable with the same name
    animal = "cat"
    print("Local:", animal)

test_scope()
print("Global:", animal)
\end{pythoncode}

Output:
\begin{pythoncode}
Local: cat
Global: dog
\end{pythoncode}

\textbf{Using the keyword global}  
It allows modifying a global variable inside a function.

\begin{pythoncode}
count = 0

def increment():
    global count
    count += 1

increment()
print(count)  # 1
\end{pythoncode}

\hrule
\subsubsection{Pandas, Series, Dataframes and selection}

Today we start with Pandas, fav library for data.

\begin{pythoncode}
    import pandas a pd #always import as pd
\end{pythoncode}

\textbf{Series} \ra a Pandas series is 1-D object that can hold any data type. Is made of 2 arrays, one for the index and the other one with the actual data.

\begin{pythoncode}
    obj = pd.Series([11, 28,72, , 5, 8])

    obj.array #information on the array
    obj.index #information on the indices
\end{pythoncode}

The left column is the index one, always. We can use custom index lists

\begin{pythoncode}
    fruits = ['apples', 'oranges', 'cherries', 'pears']
    quantities = [20, 33, 52, 10]
    S = pd.Series(quantities, index=fruits)
    #the index list is fruits

    S2 = pd.Series([17, 13, 31, 32], index=fruits)
    print(S + S2) #will sum the elements on the same position
    sum(S) #will tell us the total of the nubers in the array

    fruits2 = ['raspberries', 'oranges', 'cherries', 'pears']
    S2 = pd.Series([17, 13, 31, 32], index=fruits2)
    print(S + S2) #the operations are aligned by index, elements that don't appear in both lists will be NaN, like a Joint in relational algebra

    print(S["apples]) #we can access like a dictionary, output will be 20
    S["apples"] = 25 #to modify the element

    print(S + 2) #to add 2 to every element
    print(np.sin(S)) #to apply the sin function

    S.apply(lambda x: x if x > 25 else -1 * x) #apply will make a function to all the elements of the series, Map() will work element wise

    print(S[S > 25]) #to filter using booleans

    "apples" in S #to see if a key exists

    cities = {"London": 8615246,
    "Sesto San Giovanni": 79121,
    "Montevideo": 1405798}
    city_series = pd.Series(cities) #from dictionary to series

    print(cities.isnull()) #to see which element is NaN
    print(cities.notnull()) #to see which element is not Nan
    print(cities.fillna(0)) #to change Nan to 0.0 (floats)
    print(cities.fillna(0).astype(int)) #to change Nan to 0 (integers)

    obj.name = "Libero" #to give our series a name
    obj.index.name = "diocane" #to give the index comlumn a name
    obj.index #we see a list of the index column
\end{pythoncode}

\textbf{Characteristics of Index object}
\begin{itemize}
    \item Immutable
    \item Behaves like a fixed size set (we can check if a certain index is in the column with the in method)
    \item Can contain duplicate labels
\end{itemize}

We have several useful methods for the index objects.

\vspace{10pt}

If we put multiple series one after the other we get a matrix called \textbf{DataFrame} which can be considered like an Excel spreadsheet. Like Excel a pandas dataframe has both row and column index.

\begin{pythoncode}
    pd.concatenate([row1, row2, row3]) #to connect the rows if they share the same index
    pd.concatenate([row1, row2, row3], axis=1) #they become columns, default is 0

    df.colum.values #retrieve the columns names
    df = pd.Dataframe(df, index= a_list) #to rename the index column with the names of a list
    df = pd.Dataframe(df,
    columns=("name2", "name3", "name1") #to rearrange the columns order, same as df.reindex(["name3", "name1", "name2"])

    df.rename(columns={
    "name1" = "newname1",
    "name2" = "newname"
    }, inplace=True) #to rename columns, inplace False will return a copy of the dataset

    df = pd.Dataframe(df, columns=["name1", "name2"], index=df["other_column"]) #to put other_column as new index

    df.loc("label", "other_label") #select all the rows with that label
    df.loc(df.condition > number) #for conditions

    df.sum() #sum on all the columns
    df["column1"].sum() #sum on a single column
    df["column1"].cumsum() #cumulative sum

    #to add columns we put the new column name in the column list and we impute it with df["new_col"] = df["old_col].cumsum() for example. Default values for new columns are NaN

    df["attribute"] #access a column same as df.attribute()

    df["attribute"] = number #will fill the column with that number, for unique values use a list of the same size

    df.T #to get the transpose

    df = pd.read_csv("path/to/csv/file")

    df.head(n) #to see the first n lines of the dataframe

    df.to_csv("path/where/you/want/to/save/the/file")

    df.loc[] #access with label based approach, KeyError if it can't find the value
    df.iloc[] #access with index, IndexError if the requested index is out of bounds, except for slice indexers

    df["b": "m"] #slices can also work by labels, we will take also the middle values
    
\end{pythoncode}

How to select subsets

\begin{itemize}
    \item Fetching like a dictionary, select the index that you want
    \item Slice by index
    \item Slice by labels
    \item Fetch multiple entries
    \item Condition based selection
\end{itemize}

\vspace{10pt}

You should use .loc[] and by index

\vspace{10pt}

\textbf{iloc vs loc}

\begin{itemize}
    \item iloc = by label
    \item loc = by index
    \item duloc = lord Farquad city
\end{itemize}


\begin{pythoncode}
    df.loc["label"] #select a row by index label
    df.loc["label", ["col1", "col3"]] #row and column selection
    df.loc[:"label", ["col1", "col3"]] #as before but with slicing

    df.iloc[[2, 1]] #fetch rows with that position
    df.iloc[[1, 2], [3, 0, 1]] select certain rows and columns
    df.iloc[:, :3] #with slices
\end{pythoncode}

\subsubsection{Methods to modify views and data}

To see statistical characteristics of our dataset we can do

\begin{pythoncode}
    df.describe() #the arguments are percentiles, include and exclude
\end{pythoncode}


To see how many null values and the object in the column

\begin{pythoncode}
    df.info() #the arguments are verbose, buf, max_cols, memory_usage, null_counts
\end{pythoncode}

To see the number of unique values

\begin{pythoncode}
    df["name_col"].value_counts() #the arguments are normalize, sort, bins, dropna
\end{pythoncode}

To group or divide kind of data, basically binning

\begin{pythoncode}
    df["col1"] = df["col1"].map(lambda x: operation on x)
    df.groupby("col1") ["col2"].value_counts() #with parameters by, group_keys, dropn, as_index
\end{pythoncode}

If we want to make a multi dimensional visualization of group by we can use pivot

\begin{pythoncode}
    

pd.pivot(index="col1", columns="col2", values="col3") #with arguments data, values, index, columns, aggfunc

\end{pythoncode}

To aggregate df using a relational algebra join

\begin{pythoncode}
    df3 = pd.merge(df1, df2) #with arguments right, how, on, left_on, right_on
\end{pythoncode}

To make queries (SQL style)

\begin{pythoncode}
    df.head() #to see first 5 rows
    df.query("price < 150").head() #with argument inplace. 

    #we can apply and, or and so on, basically SQL queries inside a string
\end{pythoncode}

For 1-Hot encoding

\begin{pythoncode}
    pd.get_dummies(df["key"], prefix="key", dtype=int) #to join with another df that has those columns we use
    df = df[["data"]].join(dummies)
\end{pythoncode}

Now example of a full work with Pandas, not gonna write it.


\subsubsection{Numpy}


We use more efficient data objects than pandas. It works with multidimensional arrays. They are very pandas DF like. 
The arrays look like lists but more ordered for our computer sake. This is babsed on locality of reference.

\vspace{10pt}

Numpy arrays have a fixed size and the type is homogeneous (not if they have arrays of objects).

\vspace{10pt}

To create an array and work with them

\begin{pythoncode}
    a = np.array([1, 2, 3])
    print(type(a))
    print(a.shape)
    a[0] = 5

    b = np.array([[1, 2, 3], [4, 5, 6]])
    b[0, 0] #to access matrix like
\end{pythoncode}